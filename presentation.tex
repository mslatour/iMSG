\documentclass[11pt,a4paper,xcolor=dvipsnames]{beamer}
\usetheme{Berlin}
\usepackage[latin1]{inputenc}
\usepackage[english]{babel}
\usepackage{graphicx}

\usepackage{lmodern}

\setbeamercolor{linkbox}{fg=white,bg=blue}
\setbeamercolor{twitterbox}{fg=white,bg=Green}

\newenvironment{redenv}{\only{\setbeamercolor{local structure}{fg=red}}}{}

\author[Sander Latour]{Sander Latour\\\small{latour@uva.nl}\\\small{\texttt{@sanderlatour}}}
\institute{Universiteit van Amsterdam}
\title[Reflecting on Learning Analytics Tools]{\textbf{Reflecting on Learning Analytics Tools}\\\textit{The experiences of the Special Interest Group}}

\date{7 maart 2013}

\begin{document}

\begin{frame}
\titlepage
\end{frame}

\begin{frame}
\tableofcontents
\end{frame}

\begin{frame}
  General idea (merge zuidema with batali)
\end{frame}

\begin{frame}
  Summary of zuidema (recap)
  * including improvement points
  Iterative learning
  * general concept
  * criteria
  * components (generator, parser (parent/child))
\end{frame}

\begin{frame}
  Summary of batali (recap)
Intuition behind learning from observations
small example of a few generated observations and a derived model
  encourage / discourage
\end{frame}

\begin{frame}
  Generating observations
\end{frame}

\begin{frame}
  ViterbiX (short description of the extension)
\end{frame}

\begin{frame}
  Learning from observations using viterbix
\end{frame}

\begin{frame}
  Evaluation
\end{frame}

\begin{frame}
  Current state / Future plans
\end{frame}
\end{document}
